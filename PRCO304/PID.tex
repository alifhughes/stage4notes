\documentclass[a4paper]{article}

\usepackage{listings}
\usepackage{color}

\begin{document}

\tableofcontents

\newpage

\justify

\section{Introduction}

\paragraph{Introduction}

Set the context in very general terms
Who's the client?
    - The client is the organisation or individual who is funding the project
What is the nature of their business
    - May not business in the real sense of the terms
what is their motivation for the intended project

Currently, most music production software is a native application, that being, that it does not run on or even require the Internet. The benefits of this is that it allows to be extremely powerful, allowing the software to make full advantage of the computer's capabilities. They often have in-built synthesisers, drum machines and effects units; along with the ability of being able to plug in external machines it allows the client to produce and create any imaginable genre of music. However, even though modern applications strive to make the learning curve as minimal as possible, the use of them to a beginner can be quite daunting. This and the fact that it is near impossible to be able to collaborate on a single music project in real-time unless in the same room give rise to a few, but very important, disadvantages. \par

The rise of the Internet saw a slow and steady increase in web-based applications. Only recently with the advancements and wide acceptance of JavaScript and its in-built capabilities of creating manipulating sound has it really began to pick up. There are are a few powerful and in-depth web-based music production tools and a whole host of simple, standalone, web-based synthesisers, drum machines and effect units. However, very few, if not none, make full advantage of the raw capabilities of the internet, communication. I believe there is room for a web-based music production tool that allows for real-time collaboration. The only requirements a user needs is access to the internet, minimal knowledge of how plug in beats of a step sequencer and manipulate them (although should be intuitive enough to pick up without prior knowledge) and someone to create music with!

POTENTIALLY CHANGE TO BACKGROUND/MOTIVATION
\section{Business case}

\subsection{Background and Motivation}
Music production software has taken many forms since the birth of computing (REVISE). They allow clients to write, produce and create all different genres of music, only needing a computer and the software. Modern adaptations strive to make it as easy as possible, with little-to-none prior knowledge of music needed to get going and a wealth of in-built synthesisers and drum machines, often replicating real world tools to a high standard. However, the professional software is very costly and requires a relatively powerful computer to operate effectively on. They also are very solo artist orientated. That meaning, unless in the same room as the producer, it would be near impossible collaborate on the project. One would imagine, that the spawn of the internet would have meant a rise to many web-based applications to create music, as it did with many other industry software. However, only recently with the wider acceptance of JavaScript, with the computing power to effectively create and manipulate sound in-built, has web-based music creation applications have risen. \par

The advantages of having web-based music creation software over bespoke native applications is the same as the advantages of the Internet in general. Anyone with access to the internet, and therefore can create music. Whether it is communication of computational power, the internet allows collaboration on a world wide scale. The amount of web-based production software, synthesisers, drum machines has steadily been increasing. Most however, do not harness the true power of the internet, communication. I see that there is a real opportunity to create some music production software that allows for real-time collaboration. This meaning that not only is the access to the software is only restricted by the user having access to the internet, it also allows for two people, from anywhere in the world, to collaborate on the same music project at the same time.

\subsection{Objectives}

\begin{itemize}
    \item Two clients can simultaneously click and create music together using the step sequencer
    \item That it is available by access of a web address
    \item The UI is intuitive enough to make it possible for any skill level to compose a melody/beat
\end{itemize}

\end{document}
